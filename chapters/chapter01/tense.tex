\section{chapter 01 - Tense}\label{chapter01}

\begin{problem}{1. }
    여기가 어디입니까?

    여기가 어디인지 가르쳐 주십시오.

    저 분은 여기가 어디인지 모르는가 봅니다.
    \vspace{1pc}
\end{problem}

\begin{problem}{2. }
    [누구십니까?] [접니다.]

    누군지 가 보려무나.

    누군지 모르겠어요.
    \vspace{1pc}
\end{problem}

\begin{problem}{3. }
    그녀는 요리 솜씨가 좋다고 합니다.

    그는 배멀미를 안한다.

    나는 테니스를 잘 못한다.

    그는 말재주가 있지만, 듣는 데는 능하지 못하다.
    \vspace{1pc}
\end{problem}

\begin{problem}{4. }
    그녀는 아직 독신으로 있다.

    그들은 입을 꾹 다물고 있다.

    네 안색이 안 좋다.
    \vspace{1pc}
\end{problem}

\begin{problem}{5. }
    우리 마을은 그 호수의 동쪽에 있다.

    부산은 한국의 남부에 있다.

    아일랜드는 영국의 서쪽에 있다.
    \vspace{1pc}
\end{problem}

\begin{problem}{6. }
    교회는 동산 위에 있다.

    파리는 세느 강변에 있다.

    그 산은 우리 도의 서쪽에 있다.
    \vspace{1pc}
\end{problem}

\begin{problem}{7. }
    읍사무소 뒤에는 냇물이 있다.

    템즈 강은 런던 시 한가운데를 흐르고 있다.

    라인 강은 북해로 흘러 들어간다.
    \vspace{1pc}
\end{problem}

\begin{problem}{8. }
    이 통 속에는 포도주가 많이 있다.

    광장에 많은 사람이 있다.

    이 편지에는 오자가 둘 있다.
    \vspace{1pc}
\end{problem}

\begin{problem}{9. }
    일 년은 몇 개월입니까?

    하루는 몇 시간입니까?

    이 반에는 학생이 몇 명 있는가?
    \vspace{1pc}
\end{problem}

\begin{problem}{10. }
    [지금 몇 시입니까?] [제 시계로는 지금 9시 15분입니다.]

    당신 시계로는 지금 몇 시입니까?

    제 시계로는 정각 3시 반입니다.
    \vspace{1pc}
\end{problem}

\begin{problem}{11. }
    오늘은 맑지만 바람이 있다.

    오늘은 흐려서 매우 춥다.

    오늘은 날씨가 아주 좋다.

    하늘에는 구름 한 점 없다.
    \vspace{1pc}
\end{problem}

\begin{problem}{12. }
    오늘은 4월 13일 금요일이다.

    오늘은 무슨 요일인가?

    오늘은 일요일이라 수업이 없다.
    \vspace{1pc}
\end{problem}

\begin{problem}{13. }
    그 지방에는 눈이 많이 옵니까?

    이 지방은 따듯하고 비가 많이 온다.
    \vspace{1pc}
\end{problem}

\begin{problem}{14. }
    여기서 공항까지 (거리가) 얼마나 됩니까?

    여기서 바다까지 100미터 된다.

    너의 집에서 학교까지는 얼마나 되는가?
    \vspace{1pc}
\end{problem}

\begin{problem}{15. }
    이 방은 좀 어둡지 않습니까?

    터널 속은 아주 어둡다.

    겨울 아침은 일곱 시가 되어도 밝지가 못하다.
    \vspace{1pc}
\end{problem}

\begin{problem}{16. }
    모두들 안녕하십니까?

    (사정은) 인간도 같다.

    아기들은 어떻습니까?
    \vspace{1pc}
\end{problem}

\begin{problem}{17. }
    이 책은 표지가 연한 녹색이다.

    시계에는 시침과 분침과 초침의 셋이 있다.

    윤년에는 366일이 있다.
    \vspace{1pc}
\end{problem}

\begin{problem}{18. }
    안됐지만 돈 가진 것이 없습니다.

    가진 돈 있으면 100원만 제게 빌려 주십시오.

    나는 저 사람과는 아무 상관이 없다.
    \vspace{1pc}
\end{problem}

\begin{problem}{19. }
    그는 무거운 배낭을 짊어지고 있다.

    그는 책을 겨드랑이에 끼고 있다.

    그 여자는 왼쪽 어깨에 백을 메고 있다.
    \vspace{1pc}
\end{problem}

\begin{problem}{20. }
    그는 하루에 두 끼밖에 먹지 않습니까?

    나는 차 한 잔을 마신다.

    나는 일요일에는 점심을 안 든다.
    \vspace{1pc}
\end{problem}